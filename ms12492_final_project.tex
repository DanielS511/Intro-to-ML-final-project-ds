\documentclass[11pt]{article}

    \usepackage[breakable]{tcolorbox}
    \usepackage{parskip} % Stop auto-indenting (to mimic markdown behaviour)
    

    % Basic figure setup, for now with no caption control since it's done
    % automatically by Pandoc (which extracts ![](path) syntax from Markdown).
    \usepackage{graphicx}
    % Maintain compatibility with old templates. Remove in nbconvert 6.0
    \let\Oldincludegraphics\includegraphics
    % Ensure that by default, figures have no caption (until we provide a
    % proper Figure object with a Caption API and a way to capture that
    % in the conversion process - todo).
    \usepackage{caption}
    \DeclareCaptionFormat{nocaption}{}
    \captionsetup{format=nocaption,aboveskip=0pt,belowskip=0pt}

    \usepackage{float}
    \floatplacement{figure}{H} % forces figures to be placed at the correct location
    \usepackage{xcolor} % Allow colors to be defined
    \usepackage{enumerate} % Needed for markdown enumerations to work
    \usepackage{geometry} % Used to adjust the document margins
    \usepackage{amsmath} % Equations
    \usepackage{amssymb} % Equations
    \usepackage{textcomp} % defines textquotesingle
    % Hack from http://tex.stackexchange.com/a/47451/13684:
    \AtBeginDocument{%
        \def\PYZsq{\textquotesingle}% Upright quotes in Pygmentized code
    }
    \usepackage{upquote} % Upright quotes for verbatim code
    \usepackage{eurosym} % defines \euro

    \usepackage{iftex}
    \ifPDFTeX
        \usepackage[T1]{fontenc}
        \IfFileExists{alphabeta.sty}{
              \usepackage{alphabeta}
          }{
              \usepackage[mathletters]{ucs}
              \usepackage[utf8x]{inputenc}
          }
    \else
        \usepackage{fontspec}
        \usepackage{unicode-math}
    \fi

    \usepackage{fancyvrb} % verbatim replacement that allows latex
    \usepackage{grffile} % extends the file name processing of package graphics 
                         % to support a larger range
    \makeatletter % fix for old versions of grffile with XeLaTeX
    \@ifpackagelater{grffile}{2019/11/01}
    {
      % Do nothing on new versions
    }
    {
      \def\Gread@@xetex#1{%
        \IfFileExists{"\Gin@base".bb}%
        {\Gread@eps{\Gin@base.bb}}%
        {\Gread@@xetex@aux#1}%
      }
    }
    \makeatother
    \usepackage[Export]{adjustbox} % Used to constrain images to a maximum size
    \adjustboxset{max size={0.9\linewidth}{0.9\paperheight}}

    % The hyperref package gives us a pdf with properly built
    % internal navigation ('pdf bookmarks' for the table of contents,
    % internal cross-reference links, web links for URLs, etc.)
    \usepackage{hyperref}
    % The default LaTeX title has an obnoxious amount of whitespace. By default,
    % titling removes some of it. It also provides customization options.
    \usepackage{titling}
    \usepackage{longtable} % longtable support required by pandoc >1.10
    \usepackage{booktabs}  % table support for pandoc > 1.12.2
    \usepackage{array}     % table support for pandoc >= 2.11.3
    \usepackage{calc}      % table minipage width calculation for pandoc >= 2.11.1
    \usepackage[inline]{enumitem} % IRkernel/repr support (it uses the enumerate* environment)
    \usepackage[normalem]{ulem} % ulem is needed to support strikethroughs (\sout)
                                % normalem makes italics be italics, not underlines
    \usepackage{mathrsfs}
    

    
    % Colors for the hyperref package
    \definecolor{urlcolor}{rgb}{0,.145,.698}
    \definecolor{linkcolor}{rgb}{.71,0.21,0.01}
    \definecolor{citecolor}{rgb}{.12,.54,.11}

    % ANSI colors
    \definecolor{ansi-black}{HTML}{3E424D}
    \definecolor{ansi-black-intense}{HTML}{282C36}
    \definecolor{ansi-red}{HTML}{E75C58}
    \definecolor{ansi-red-intense}{HTML}{B22B31}
    \definecolor{ansi-green}{HTML}{00A250}
    \definecolor{ansi-green-intense}{HTML}{007427}
    \definecolor{ansi-yellow}{HTML}{DDB62B}
    \definecolor{ansi-yellow-intense}{HTML}{B27D12}
    \definecolor{ansi-blue}{HTML}{208FFB}
    \definecolor{ansi-blue-intense}{HTML}{0065CA}
    \definecolor{ansi-magenta}{HTML}{D160C4}
    \definecolor{ansi-magenta-intense}{HTML}{A03196}
    \definecolor{ansi-cyan}{HTML}{60C6C8}
    \definecolor{ansi-cyan-intense}{HTML}{258F8F}
    \definecolor{ansi-white}{HTML}{C5C1B4}
    \definecolor{ansi-white-intense}{HTML}{A1A6B2}
    \definecolor{ansi-default-inverse-fg}{HTML}{FFFFFF}
    \definecolor{ansi-default-inverse-bg}{HTML}{000000}

    % common color for the border for error outputs.
    \definecolor{outerrorbackground}{HTML}{FFDFDF}

    % commands and environments needed by pandoc snippets
    % extracted from the output of `pandoc -s`
    \providecommand{\tightlist}{%
      \setlength{\itemsep}{0pt}\setlength{\parskip}{0pt}}
    \DefineVerbatimEnvironment{Highlighting}{Verbatim}{commandchars=\\\{\}}
    % Add ',fontsize=\small' for more characters per line
    \newenvironment{Shaded}{}{}
    \newcommand{\KeywordTok}[1]{\textcolor[rgb]{0.00,0.44,0.13}{\textbf{{#1}}}}
    \newcommand{\DataTypeTok}[1]{\textcolor[rgb]{0.56,0.13,0.00}{{#1}}}
    \newcommand{\DecValTok}[1]{\textcolor[rgb]{0.25,0.63,0.44}{{#1}}}
    \newcommand{\BaseNTok}[1]{\textcolor[rgb]{0.25,0.63,0.44}{{#1}}}
    \newcommand{\FloatTok}[1]{\textcolor[rgb]{0.25,0.63,0.44}{{#1}}}
    \newcommand{\CharTok}[1]{\textcolor[rgb]{0.25,0.44,0.63}{{#1}}}
    \newcommand{\StringTok}[1]{\textcolor[rgb]{0.25,0.44,0.63}{{#1}}}
    \newcommand{\CommentTok}[1]{\textcolor[rgb]{0.38,0.63,0.69}{\textit{{#1}}}}
    \newcommand{\OtherTok}[1]{\textcolor[rgb]{0.00,0.44,0.13}{{#1}}}
    \newcommand{\AlertTok}[1]{\textcolor[rgb]{1.00,0.00,0.00}{\textbf{{#1}}}}
    \newcommand{\FunctionTok}[1]{\textcolor[rgb]{0.02,0.16,0.49}{{#1}}}
    \newcommand{\RegionMarkerTok}[1]{{#1}}
    \newcommand{\ErrorTok}[1]{\textcolor[rgb]{1.00,0.00,0.00}{\textbf{{#1}}}}
    \newcommand{\NormalTok}[1]{{#1}}
    
    % Additional commands for more recent versions of Pandoc
    \newcommand{\ConstantTok}[1]{\textcolor[rgb]{0.53,0.00,0.00}{{#1}}}
    \newcommand{\SpecialCharTok}[1]{\textcolor[rgb]{0.25,0.44,0.63}{{#1}}}
    \newcommand{\VerbatimStringTok}[1]{\textcolor[rgb]{0.25,0.44,0.63}{{#1}}}
    \newcommand{\SpecialStringTok}[1]{\textcolor[rgb]{0.73,0.40,0.53}{{#1}}}
    \newcommand{\ImportTok}[1]{{#1}}
    \newcommand{\DocumentationTok}[1]{\textcolor[rgb]{0.73,0.13,0.13}{\textit{{#1}}}}
    \newcommand{\AnnotationTok}[1]{\textcolor[rgb]{0.38,0.63,0.69}{\textbf{\textit{{#1}}}}}
    \newcommand{\CommentVarTok}[1]{\textcolor[rgb]{0.38,0.63,0.69}{\textbf{\textit{{#1}}}}}
    \newcommand{\VariableTok}[1]{\textcolor[rgb]{0.10,0.09,0.49}{{#1}}}
    \newcommand{\ControlFlowTok}[1]{\textcolor[rgb]{0.00,0.44,0.13}{\textbf{{#1}}}}
    \newcommand{\OperatorTok}[1]{\textcolor[rgb]{0.40,0.40,0.40}{{#1}}}
    \newcommand{\BuiltInTok}[1]{{#1}}
    \newcommand{\ExtensionTok}[1]{{#1}}
    \newcommand{\PreprocessorTok}[1]{\textcolor[rgb]{0.74,0.48,0.00}{{#1}}}
    \newcommand{\AttributeTok}[1]{\textcolor[rgb]{0.49,0.56,0.16}{{#1}}}
    \newcommand{\InformationTok}[1]{\textcolor[rgb]{0.38,0.63,0.69}{\textbf{\textit{{#1}}}}}
    \newcommand{\WarningTok}[1]{\textcolor[rgb]{0.38,0.63,0.69}{\textbf{\textit{{#1}}}}}
    
    
    % Define a nice break command that doesn't care if a line doesn't already
    % exist.
    \def\br{\hspace*{\fill} \\* }
    % Math Jax compatibility definitions
    \def\gt{>}
    \def\lt{<}
    \let\Oldtex\TeX
    \let\Oldlatex\LaTeX
    \renewcommand{\TeX}{\textrm{\Oldtex}}
    \renewcommand{\LaTeX}{\textrm{\Oldlatex}}
    % Document parameters
    % Document title
    \title{ms12492\_final\_project}
    
    
    
    
    
% Pygments definitions
\makeatletter
\def\PY@reset{\let\PY@it=\relax \let\PY@bf=\relax%
    \let\PY@ul=\relax \let\PY@tc=\relax%
    \let\PY@bc=\relax \let\PY@ff=\relax}
\def\PY@tok#1{\csname PY@tok@#1\endcsname}
\def\PY@toks#1+{\ifx\relax#1\empty\else%
    \PY@tok{#1}\expandafter\PY@toks\fi}
\def\PY@do#1{\PY@bc{\PY@tc{\PY@ul{%
    \PY@it{\PY@bf{\PY@ff{#1}}}}}}}
\def\PY#1#2{\PY@reset\PY@toks#1+\relax+\PY@do{#2}}

\@namedef{PY@tok@w}{\def\PY@tc##1{\textcolor[rgb]{0.73,0.73,0.73}{##1}}}
\@namedef{PY@tok@c}{\let\PY@it=\textit\def\PY@tc##1{\textcolor[rgb]{0.24,0.48,0.48}{##1}}}
\@namedef{PY@tok@cp}{\def\PY@tc##1{\textcolor[rgb]{0.61,0.40,0.00}{##1}}}
\@namedef{PY@tok@k}{\let\PY@bf=\textbf\def\PY@tc##1{\textcolor[rgb]{0.00,0.50,0.00}{##1}}}
\@namedef{PY@tok@kp}{\def\PY@tc##1{\textcolor[rgb]{0.00,0.50,0.00}{##1}}}
\@namedef{PY@tok@kt}{\def\PY@tc##1{\textcolor[rgb]{0.69,0.00,0.25}{##1}}}
\@namedef{PY@tok@o}{\def\PY@tc##1{\textcolor[rgb]{0.40,0.40,0.40}{##1}}}
\@namedef{PY@tok@ow}{\let\PY@bf=\textbf\def\PY@tc##1{\textcolor[rgb]{0.67,0.13,1.00}{##1}}}
\@namedef{PY@tok@nb}{\def\PY@tc##1{\textcolor[rgb]{0.00,0.50,0.00}{##1}}}
\@namedef{PY@tok@nf}{\def\PY@tc##1{\textcolor[rgb]{0.00,0.00,1.00}{##1}}}
\@namedef{PY@tok@nc}{\let\PY@bf=\textbf\def\PY@tc##1{\textcolor[rgb]{0.00,0.00,1.00}{##1}}}
\@namedef{PY@tok@nn}{\let\PY@bf=\textbf\def\PY@tc##1{\textcolor[rgb]{0.00,0.00,1.00}{##1}}}
\@namedef{PY@tok@ne}{\let\PY@bf=\textbf\def\PY@tc##1{\textcolor[rgb]{0.80,0.25,0.22}{##1}}}
\@namedef{PY@tok@nv}{\def\PY@tc##1{\textcolor[rgb]{0.10,0.09,0.49}{##1}}}
\@namedef{PY@tok@no}{\def\PY@tc##1{\textcolor[rgb]{0.53,0.00,0.00}{##1}}}
\@namedef{PY@tok@nl}{\def\PY@tc##1{\textcolor[rgb]{0.46,0.46,0.00}{##1}}}
\@namedef{PY@tok@ni}{\let\PY@bf=\textbf\def\PY@tc##1{\textcolor[rgb]{0.44,0.44,0.44}{##1}}}
\@namedef{PY@tok@na}{\def\PY@tc##1{\textcolor[rgb]{0.41,0.47,0.13}{##1}}}
\@namedef{PY@tok@nt}{\let\PY@bf=\textbf\def\PY@tc##1{\textcolor[rgb]{0.00,0.50,0.00}{##1}}}
\@namedef{PY@tok@nd}{\def\PY@tc##1{\textcolor[rgb]{0.67,0.13,1.00}{##1}}}
\@namedef{PY@tok@s}{\def\PY@tc##1{\textcolor[rgb]{0.73,0.13,0.13}{##1}}}
\@namedef{PY@tok@sd}{\let\PY@it=\textit\def\PY@tc##1{\textcolor[rgb]{0.73,0.13,0.13}{##1}}}
\@namedef{PY@tok@si}{\let\PY@bf=\textbf\def\PY@tc##1{\textcolor[rgb]{0.64,0.35,0.47}{##1}}}
\@namedef{PY@tok@se}{\let\PY@bf=\textbf\def\PY@tc##1{\textcolor[rgb]{0.67,0.36,0.12}{##1}}}
\@namedef{PY@tok@sr}{\def\PY@tc##1{\textcolor[rgb]{0.64,0.35,0.47}{##1}}}
\@namedef{PY@tok@ss}{\def\PY@tc##1{\textcolor[rgb]{0.10,0.09,0.49}{##1}}}
\@namedef{PY@tok@sx}{\def\PY@tc##1{\textcolor[rgb]{0.00,0.50,0.00}{##1}}}
\@namedef{PY@tok@m}{\def\PY@tc##1{\textcolor[rgb]{0.40,0.40,0.40}{##1}}}
\@namedef{PY@tok@gh}{\let\PY@bf=\textbf\def\PY@tc##1{\textcolor[rgb]{0.00,0.00,0.50}{##1}}}
\@namedef{PY@tok@gu}{\let\PY@bf=\textbf\def\PY@tc##1{\textcolor[rgb]{0.50,0.00,0.50}{##1}}}
\@namedef{PY@tok@gd}{\def\PY@tc##1{\textcolor[rgb]{0.63,0.00,0.00}{##1}}}
\@namedef{PY@tok@gi}{\def\PY@tc##1{\textcolor[rgb]{0.00,0.52,0.00}{##1}}}
\@namedef{PY@tok@gr}{\def\PY@tc##1{\textcolor[rgb]{0.89,0.00,0.00}{##1}}}
\@namedef{PY@tok@ge}{\let\PY@it=\textit}
\@namedef{PY@tok@gs}{\let\PY@bf=\textbf}
\@namedef{PY@tok@gp}{\let\PY@bf=\textbf\def\PY@tc##1{\textcolor[rgb]{0.00,0.00,0.50}{##1}}}
\@namedef{PY@tok@go}{\def\PY@tc##1{\textcolor[rgb]{0.44,0.44,0.44}{##1}}}
\@namedef{PY@tok@gt}{\def\PY@tc##1{\textcolor[rgb]{0.00,0.27,0.87}{##1}}}
\@namedef{PY@tok@err}{\def\PY@bc##1{{\setlength{\fboxsep}{\string -\fboxrule}\fcolorbox[rgb]{1.00,0.00,0.00}{1,1,1}{\strut ##1}}}}
\@namedef{PY@tok@kc}{\let\PY@bf=\textbf\def\PY@tc##1{\textcolor[rgb]{0.00,0.50,0.00}{##1}}}
\@namedef{PY@tok@kd}{\let\PY@bf=\textbf\def\PY@tc##1{\textcolor[rgb]{0.00,0.50,0.00}{##1}}}
\@namedef{PY@tok@kn}{\let\PY@bf=\textbf\def\PY@tc##1{\textcolor[rgb]{0.00,0.50,0.00}{##1}}}
\@namedef{PY@tok@kr}{\let\PY@bf=\textbf\def\PY@tc##1{\textcolor[rgb]{0.00,0.50,0.00}{##1}}}
\@namedef{PY@tok@bp}{\def\PY@tc##1{\textcolor[rgb]{0.00,0.50,0.00}{##1}}}
\@namedef{PY@tok@fm}{\def\PY@tc##1{\textcolor[rgb]{0.00,0.00,1.00}{##1}}}
\@namedef{PY@tok@vc}{\def\PY@tc##1{\textcolor[rgb]{0.10,0.09,0.49}{##1}}}
\@namedef{PY@tok@vg}{\def\PY@tc##1{\textcolor[rgb]{0.10,0.09,0.49}{##1}}}
\@namedef{PY@tok@vi}{\def\PY@tc##1{\textcolor[rgb]{0.10,0.09,0.49}{##1}}}
\@namedef{PY@tok@vm}{\def\PY@tc##1{\textcolor[rgb]{0.10,0.09,0.49}{##1}}}
\@namedef{PY@tok@sa}{\def\PY@tc##1{\textcolor[rgb]{0.73,0.13,0.13}{##1}}}
\@namedef{PY@tok@sb}{\def\PY@tc##1{\textcolor[rgb]{0.73,0.13,0.13}{##1}}}
\@namedef{PY@tok@sc}{\def\PY@tc##1{\textcolor[rgb]{0.73,0.13,0.13}{##1}}}
\@namedef{PY@tok@dl}{\def\PY@tc##1{\textcolor[rgb]{0.73,0.13,0.13}{##1}}}
\@namedef{PY@tok@s2}{\def\PY@tc##1{\textcolor[rgb]{0.73,0.13,0.13}{##1}}}
\@namedef{PY@tok@sh}{\def\PY@tc##1{\textcolor[rgb]{0.73,0.13,0.13}{##1}}}
\@namedef{PY@tok@s1}{\def\PY@tc##1{\textcolor[rgb]{0.73,0.13,0.13}{##1}}}
\@namedef{PY@tok@mb}{\def\PY@tc##1{\textcolor[rgb]{0.40,0.40,0.40}{##1}}}
\@namedef{PY@tok@mf}{\def\PY@tc##1{\textcolor[rgb]{0.40,0.40,0.40}{##1}}}
\@namedef{PY@tok@mh}{\def\PY@tc##1{\textcolor[rgb]{0.40,0.40,0.40}{##1}}}
\@namedef{PY@tok@mi}{\def\PY@tc##1{\textcolor[rgb]{0.40,0.40,0.40}{##1}}}
\@namedef{PY@tok@il}{\def\PY@tc##1{\textcolor[rgb]{0.40,0.40,0.40}{##1}}}
\@namedef{PY@tok@mo}{\def\PY@tc##1{\textcolor[rgb]{0.40,0.40,0.40}{##1}}}
\@namedef{PY@tok@ch}{\let\PY@it=\textit\def\PY@tc##1{\textcolor[rgb]{0.24,0.48,0.48}{##1}}}
\@namedef{PY@tok@cm}{\let\PY@it=\textit\def\PY@tc##1{\textcolor[rgb]{0.24,0.48,0.48}{##1}}}
\@namedef{PY@tok@cpf}{\let\PY@it=\textit\def\PY@tc##1{\textcolor[rgb]{0.24,0.48,0.48}{##1}}}
\@namedef{PY@tok@c1}{\let\PY@it=\textit\def\PY@tc##1{\textcolor[rgb]{0.24,0.48,0.48}{##1}}}
\@namedef{PY@tok@cs}{\let\PY@it=\textit\def\PY@tc##1{\textcolor[rgb]{0.24,0.48,0.48}{##1}}}

\def\PYZbs{\char`\\}
\def\PYZus{\char`\_}
\def\PYZob{\char`\{}
\def\PYZcb{\char`\}}
\def\PYZca{\char`\^}
\def\PYZam{\char`\&}
\def\PYZlt{\char`\<}
\def\PYZgt{\char`\>}
\def\PYZsh{\char`\#}
\def\PYZpc{\char`\%}
\def\PYZdl{\char`\$}
\def\PYZhy{\char`\-}
\def\PYZsq{\char`\'}
\def\PYZdq{\char`\"}
\def\PYZti{\char`\~}
% for compatibility with earlier versions
\def\PYZat{@}
\def\PYZlb{[}
\def\PYZrb{]}
\makeatother


    % For linebreaks inside Verbatim environment from package fancyvrb. 
    \makeatletter
        \newbox\Wrappedcontinuationbox 
        \newbox\Wrappedvisiblespacebox 
        \newcommand*\Wrappedvisiblespace {\textcolor{red}{\textvisiblespace}} 
        \newcommand*\Wrappedcontinuationsymbol {\textcolor{red}{\llap{\tiny$\m@th\hookrightarrow$}}} 
        \newcommand*\Wrappedcontinuationindent {3ex } 
        \newcommand*\Wrappedafterbreak {\kern\Wrappedcontinuationindent\copy\Wrappedcontinuationbox} 
        % Take advantage of the already applied Pygments mark-up to insert 
        % potential linebreaks for TeX processing. 
        %        {, <, #, %, $, ' and ": go to next line. 
        %        _, }, ^, &, >, - and ~: stay at end of broken line. 
        % Use of \textquotesingle for straight quote. 
        \newcommand*\Wrappedbreaksatspecials {% 
            \def\PYGZus{\discretionary{\char`\_}{\Wrappedafterbreak}{\char`\_}}% 
            \def\PYGZob{\discretionary{}{\Wrappedafterbreak\char`\{}{\char`\{}}% 
            \def\PYGZcb{\discretionary{\char`\}}{\Wrappedafterbreak}{\char`\}}}% 
            \def\PYGZca{\discretionary{\char`\^}{\Wrappedafterbreak}{\char`\^}}% 
            \def\PYGZam{\discretionary{\char`\&}{\Wrappedafterbreak}{\char`\&}}% 
            \def\PYGZlt{\discretionary{}{\Wrappedafterbreak\char`\<}{\char`\<}}% 
            \def\PYGZgt{\discretionary{\char`\>}{\Wrappedafterbreak}{\char`\>}}% 
            \def\PYGZsh{\discretionary{}{\Wrappedafterbreak\char`\#}{\char`\#}}% 
            \def\PYGZpc{\discretionary{}{\Wrappedafterbreak\char`\%}{\char`\%}}% 
            \def\PYGZdl{\discretionary{}{\Wrappedafterbreak\char`\$}{\char`\$}}% 
            \def\PYGZhy{\discretionary{\char`\-}{\Wrappedafterbreak}{\char`\-}}% 
            \def\PYGZsq{\discretionary{}{\Wrappedafterbreak\textquotesingle}{\textquotesingle}}% 
            \def\PYGZdq{\discretionary{}{\Wrappedafterbreak\char`\"}{\char`\"}}% 
            \def\PYGZti{\discretionary{\char`\~}{\Wrappedafterbreak}{\char`\~}}% 
        } 
        % Some characters . , ; ? ! / are not pygmentized. 
        % This macro makes them "active" and they will insert potential linebreaks 
        \newcommand*\Wrappedbreaksatpunct {% 
            \lccode`\~`\.\lowercase{\def~}{\discretionary{\hbox{\char`\.}}{\Wrappedafterbreak}{\hbox{\char`\.}}}% 
            \lccode`\~`\,\lowercase{\def~}{\discretionary{\hbox{\char`\,}}{\Wrappedafterbreak}{\hbox{\char`\,}}}% 
            \lccode`\~`\;\lowercase{\def~}{\discretionary{\hbox{\char`\;}}{\Wrappedafterbreak}{\hbox{\char`\;}}}% 
            \lccode`\~`\:\lowercase{\def~}{\discretionary{\hbox{\char`\:}}{\Wrappedafterbreak}{\hbox{\char`\:}}}% 
            \lccode`\~`\?\lowercase{\def~}{\discretionary{\hbox{\char`\?}}{\Wrappedafterbreak}{\hbox{\char`\?}}}% 
            \lccode`\~`\!\lowercase{\def~}{\discretionary{\hbox{\char`\!}}{\Wrappedafterbreak}{\hbox{\char`\!}}}% 
            \lccode`\~`\/\lowercase{\def~}{\discretionary{\hbox{\char`\/}}{\Wrappedafterbreak}{\hbox{\char`\/}}}% 
            \catcode`\.\active
            \catcode`\,\active 
            \catcode`\;\active
            \catcode`\:\active
            \catcode`\?\active
            \catcode`\!\active
            \catcode`\/\active 
            \lccode`\~`\~ 	
        }
    \makeatother

    \let\OriginalVerbatim=\Verbatim
    \makeatletter
    \renewcommand{\Verbatim}[1][1]{%
        %\parskip\z@skip
        \sbox\Wrappedcontinuationbox {\Wrappedcontinuationsymbol}%
        \sbox\Wrappedvisiblespacebox {\FV@SetupFont\Wrappedvisiblespace}%
        \def\FancyVerbFormatLine ##1{\hsize\linewidth
            \vtop{\raggedright\hyphenpenalty\z@\exhyphenpenalty\z@
                \doublehyphendemerits\z@\finalhyphendemerits\z@
                \strut ##1\strut}%
        }%
        % If the linebreak is at a space, the latter will be displayed as visible
        % space at end of first line, and a continuation symbol starts next line.
        % Stretch/shrink are however usually zero for typewriter font.
        \def\FV@Space {%
            \nobreak\hskip\z@ plus\fontdimen3\font minus\fontdimen4\font
            \discretionary{\copy\Wrappedvisiblespacebox}{\Wrappedafterbreak}
            {\kern\fontdimen2\font}%
        }%
        
        % Allow breaks at special characters using \PYG... macros.
        \Wrappedbreaksatspecials
        % Breaks at punctuation characters . , ; ? ! and / need catcode=\active 	
        \OriginalVerbatim[#1,codes*=\Wrappedbreaksatpunct]%
    }
    \makeatother

    % Exact colors from NB
    \definecolor{incolor}{HTML}{303F9F}
    \definecolor{outcolor}{HTML}{D84315}
    \definecolor{cellborder}{HTML}{CFCFCF}
    \definecolor{cellbackground}{HTML}{F7F7F7}
    
    % prompt
    \makeatletter
    \newcommand{\boxspacing}{\kern\kvtcb@left@rule\kern\kvtcb@boxsep}
    \makeatother
    \newcommand{\prompt}[4]{
        {\ttfamily\llap{{\color{#2}[#3]:\hspace{3pt}#4}}\vspace{-\baselineskip}}
    }
    

    
    % Prevent overflowing lines due to hard-to-break entities
    \sloppy 
    % Setup hyperref package
    \hypersetup{
      breaklinks=true,  % so long urls are correctly broken across lines
      colorlinks=true,
      urlcolor=urlcolor,
      linkcolor=linkcolor,
      citecolor=citecolor,
      }
    % Slightly bigger margins than the latex defaults
    
    \geometry{verbose,tmargin=1in,bmargin=1in,lmargin=1in,rmargin=1in}
    
    

\begin{document}
    
    \maketitle
    
    

    
    \hypertarget{method}{%
\section{Method}\label{method}}

    \hypertarget{import-packages}{%
\subsection{Import Packages}\label{import-packages}}

    \begin{tcolorbox}[breakable, size=fbox, boxrule=1pt, pad at break*=1mm,colback=cellbackground, colframe=cellborder]
\prompt{In}{incolor}{1}{\boxspacing}
\begin{Verbatim}[commandchars=\\\{\}]
\PY{k+kn}{import} \PY{n+nn}{torch}
\PY{k+kn}{import} \PY{n+nn}{torch}\PY{n+nn}{.}\PY{n+nn}{nn} \PY{k}{as} \PY{n+nn}{nn}
\PY{k+kn}{from} \PY{n+nn}{torch}\PY{n+nn}{.}\PY{n+nn}{utils}\PY{n+nn}{.}\PY{n+nn}{data} \PY{k+kn}{import} \PY{n}{Dataset}
\PY{k+kn}{from} \PY{n+nn}{torchvision} \PY{k+kn}{import} \PY{n}{transforms}
\PY{k+kn}{import} \PY{n+nn}{numpy} \PY{k}{as} \PY{n+nn}{np}
\PY{k+kn}{import} \PY{n+nn}{os}
\PY{k+kn}{import} \PY{n+nn}{cv2}
\PY{k+kn}{import} \PY{n+nn}{pickle} \PY{k}{as} \PY{n+nn}{pkl}
\PY{k+kn}{from} \PY{n+nn}{torchvision}\PY{n+nn}{.}\PY{n+nn}{models} \PY{k+kn}{import} \PY{n}{resnet50}\PY{p}{,} \PY{n}{ResNet50\PYZus{}Weights}
\end{Verbatim}
\end{tcolorbox}

    \hypertarget{define-device-to-use}{%
\subsection{Define Device to use}\label{define-device-to-use}}

    \begin{tcolorbox}[breakable, size=fbox, boxrule=1pt, pad at break*=1mm,colback=cellbackground, colframe=cellborder]
\prompt{In}{incolor}{2}{\boxspacing}
\begin{Verbatim}[commandchars=\\\{\}]
\PY{n}{device} \PY{o}{=} \PY{n}{torch}\PY{o}{.}\PY{n}{device}\PY{p}{(}\PY{l+s+s2}{\PYZdq{}}\PY{l+s+s2}{cuda}\PY{l+s+s2}{\PYZdq{}} \PY{k}{if} \PY{n}{torch}\PY{o}{.}\PY{n}{cuda}\PY{o}{.}\PY{n}{is\PYZus{}available}\PY{p}{(}\PY{p}{)} \PY{k}{else} \PY{l+s+s2}{\PYZdq{}}\PY{l+s+s2}{cpu}\PY{l+s+s2}{\PYZdq{}}\PY{p}{)}
\end{Verbatim}
\end{tcolorbox}

    \hypertarget{load-and-normalize-data}{%
\subsection{Load and Normalize Data}\label{load-and-normalize-data}}

    \hypertarget{define-lazy-loader-class}{%
\subsubsection{Define Lazy Loader
Class}\label{define-lazy-loader-class}}

We only do transformation and normalization on the imgs. Doing things on
depth seems to have negative effect on result.

    \begin{tcolorbox}[breakable, size=fbox, boxrule=1pt, pad at break*=1mm,colback=cellbackground, colframe=cellborder]
\prompt{In}{incolor}{3}{\boxspacing}
\begin{Verbatim}[commandchars=\\\{\}]
\PY{k}{class} \PY{n+nc}{LazyLoadDataset}\PY{p}{(}\PY{n}{Dataset}\PY{p}{)}\PY{p}{:}
  \PY{l+s+sd}{\PYZdq{}\PYZdq{}\PYZdq{}}
\PY{l+s+sd}{    Class for lazy loading }
\PY{l+s+sd}{    }
\PY{l+s+sd}{    Constructor:}
\PY{l+s+sd}{      path(String): the path in which lazy data are located}
\PY{l+s+sd}{      train(Boolean): the data is for train or not}
\PY{l+s+sd}{      transform(function): the function for us to perform transform on images}
\PY{l+s+sd}{  \PYZdq{}\PYZdq{}\PYZdq{}}
  \PY{k}{def} \PY{n+nf+fm}{\PYZus{}\PYZus{}init\PYZus{}\PYZus{}}\PY{p}{(}\PY{n+nb+bp}{self}\PY{p}{,} \PY{n}{path}\PY{p}{,} \PY{n}{train}\PY{o}{=}\PY{k+kc}{True}\PY{p}{,} \PY{n}{transform}\PY{o}{=}\PY{k+kc}{None}\PY{p}{)}\PY{p}{:}
    \PY{n+nb+bp}{self}\PY{o}{.}\PY{n}{transform} \PY{o}{=} \PY{n}{transform}
    \PY{n}{path} \PY{o}{=} \PY{n}{path} \PY{o}{+} \PY{p}{(}\PY{l+s+s2}{\PYZdq{}}\PY{l+s+s2}{train/}\PY{l+s+s2}{\PYZdq{}} \PY{k}{if} \PY{n}{train} \PY{k}{else} \PY{l+s+s2}{\PYZdq{}}\PY{l+s+s2}{test/}\PY{l+s+s2}{\PYZdq{}}\PY{p}{)}

    \PY{n+nb+bp}{self}\PY{o}{.}\PY{n}{pathX} \PY{o}{=} \PY{n}{path} \PY{o}{+} \PY{l+s+s2}{\PYZdq{}}\PY{l+s+s2}{X/}\PY{l+s+s2}{\PYZdq{}}
    \PY{n+nb+bp}{self}\PY{o}{.}\PY{n}{pathY} \PY{o}{=} \PY{n}{path} \PY{o}{+} \PY{l+s+s2}{\PYZdq{}}\PY{l+s+s2}{Y/}\PY{l+s+s2}{\PYZdq{}}

    \PY{n+nb+bp}{self}\PY{o}{.}\PY{n}{data} \PY{o}{=} \PY{n}{os}\PY{o}{.}\PY{n}{listdir}\PY{p}{(}\PY{n+nb+bp}{self}\PY{o}{.}\PY{n}{pathX}\PY{p}{)}
    \PY{n+nb+bp}{self}\PY{o}{.}\PY{n}{train} \PY{o}{=} \PY{n}{train}
  
  \PY{k}{def} \PY{n+nf+fm}{\PYZus{}\PYZus{}getitem\PYZus{}\PYZus{}}\PY{p}{(}\PY{n+nb+bp}{self}\PY{p}{,} \PY{n}{idx}\PY{p}{)}\PY{p}{:}
    \PY{n}{f} \PY{o}{=} \PY{n+nb+bp}{self}\PY{o}{.}\PY{n}{data}\PY{p}{[}\PY{n}{idx}\PY{p}{]}

    \PY{n}{img0} \PY{o}{=} \PY{n}{cv2}\PY{o}{.}\PY{n}{imread}\PY{p}{(}\PY{n+nb+bp}{self}\PY{o}{.}\PY{n}{pathX} \PY{o}{+} \PY{n}{f} \PY{o}{+} \PY{l+s+s2}{\PYZdq{}}\PY{l+s+s2}{/rgb/0.png}\PY{l+s+s2}{\PYZdq{}}\PY{p}{)}
    \PY{n}{img1} \PY{o}{=} \PY{n}{cv2}\PY{o}{.}\PY{n}{imread}\PY{p}{(}\PY{n+nb+bp}{self}\PY{o}{.}\PY{n}{pathX} \PY{o}{+} \PY{n}{f} \PY{o}{+} \PY{l+s+s2}{\PYZdq{}}\PY{l+s+s2}{/rgb/1.png}\PY{l+s+s2}{\PYZdq{}}\PY{p}{)}
    \PY{n}{img2} \PY{o}{=} \PY{n}{cv2}\PY{o}{.}\PY{n}{imread}\PY{p}{(}\PY{n+nb+bp}{self}\PY{o}{.}\PY{n}{pathX} \PY{o}{+} \PY{n}{f} \PY{o}{+} \PY{l+s+s2}{\PYZdq{}}\PY{l+s+s2}{/rgb/2.png}\PY{l+s+s2}{\PYZdq{}}\PY{p}{)}

    \PY{k}{if} \PY{n+nb+bp}{self}\PY{o}{.}\PY{n}{transform} \PY{o+ow}{is} \PY{o+ow}{not} \PY{k+kc}{None}\PY{p}{:}
      \PY{n}{img0} \PY{o}{=} \PY{n+nb+bp}{self}\PY{o}{.}\PY{n}{transform}\PY{p}{(}\PY{n}{img0}\PY{p}{)}
      \PY{n}{img1} \PY{o}{=} \PY{n+nb+bp}{self}\PY{o}{.}\PY{n}{transform}\PY{p}{(}\PY{n}{img1}\PY{p}{)}
      \PY{n}{img2} \PY{o}{=} \PY{n+nb+bp}{self}\PY{o}{.}\PY{n}{transform}\PY{p}{(}\PY{n}{img2}\PY{p}{)}
    
    \PY{n}{depth} \PY{o}{=} \PY{n}{np}\PY{o}{.}\PY{n}{load}\PY{p}{(}\PY{n+nb+bp}{self}\PY{o}{.}\PY{n}{pathX} \PY{o}{+} \PY{n}{f} \PY{o}{+} \PY{l+s+s2}{\PYZdq{}}\PY{l+s+s2}{/depth.npy}\PY{l+s+s2}{\PYZdq{}}\PY{p}{)}
    \PY{c+c1}{\PYZsh{} depth = self.transform(depth)}

    \PY{n}{field\PYZus{}id} \PY{o}{=} \PY{n}{pkl}\PY{o}{.}\PY{n}{load}\PY{p}{(}\PY{n+nb}{open}\PY{p}{(}\PY{n+nb+bp}{self}\PY{o}{.}\PY{n}{pathX} \PY{o}{+} \PY{n}{f} \PY{o}{+} \PY{l+s+s2}{\PYZdq{}}\PY{l+s+s2}{/field\PYZus{}id.pkl}\PY{l+s+s2}{\PYZdq{}}\PY{p}{,} \PY{l+s+s2}{\PYZdq{}}\PY{l+s+s2}{rb}\PY{l+s+s2}{\PYZdq{}}\PY{p}{)}\PY{p}{)}

    \PY{k}{if} \PY{n+nb+bp}{self}\PY{o}{.}\PY{n}{train}\PY{p}{:}
      \PY{n}{Y} \PY{o}{=} \PY{n}{np}\PY{o}{.}\PY{n}{load}\PY{p}{(}\PY{n+nb+bp}{self}\PY{o}{.}\PY{n}{pathY} \PY{o}{+} \PY{n}{f} \PY{o}{+} \PY{l+s+s2}{\PYZdq{}}\PY{l+s+s2}{.npy}\PY{l+s+s2}{\PYZdq{}}\PY{p}{)} \PY{o}{*} \PY{l+m+mi}{1000}
      \PY{k}{return} \PY{p}{(}\PY{n}{img0}\PY{p}{,} \PY{n}{img1}\PY{p}{,} \PY{n}{img2}\PY{p}{,} \PY{n}{depth}\PY{p}{)}\PY{p}{,} \PY{n}{Y}
    \PY{k}{else}\PY{p}{:}
      \PY{k}{return} \PY{p}{(}\PY{n}{img0}\PY{p}{,} \PY{n}{img1}\PY{p}{,} \PY{n}{img2}\PY{p}{,} \PY{n}{depth}\PY{p}{)}

  \PY{k}{def} \PY{n+nf+fm}{\PYZus{}\PYZus{}len\PYZus{}\PYZus{}}\PY{p}{(}\PY{n+nb+bp}{self}\PY{p}{)}\PY{p}{:}
    \PY{k}{return} \PY{n+nb}{len}\PY{p}{(}\PY{n+nb+bp}{self}\PY{o}{.}\PY{n}{data}\PY{p}{)}
\end{Verbatim}
\end{tcolorbox}

    \hypertarget{define-transformation-and-normalization-functions}{%
\subsubsection{Define transformation and normalization
functions}\label{define-transformation-and-normalization-functions}}

    Here we use \texttt{transform} from \texttt{torchvision} to do
transformation and normalization.

    \begin{tcolorbox}[breakable, size=fbox, boxrule=1pt, pad at break*=1mm,colback=cellbackground, colframe=cellborder]
\prompt{In}{incolor}{4}{\boxspacing}
\begin{Verbatim}[commandchars=\\\{\}]
\PY{n}{data\PYZus{}transforms} \PY{o}{=} \PY{p}{\PYZob{}}
    \PY{l+s+s1}{\PYZsq{}}\PY{l+s+s1}{train}\PY{l+s+s1}{\PYZsq{}}\PY{p}{:} \PY{n}{transforms}\PY{o}{.}\PY{n}{Compose}\PY{p}{(}\PY{p}{[}
        \PY{n}{transforms}\PY{o}{.}\PY{n}{ToPILImage}\PY{p}{(}\PY{p}{)}\PY{p}{,}
        \PY{n}{transforms}\PY{o}{.}\PY{n}{RandomRotation}\PY{p}{(}\PY{l+m+mi}{45}\PY{p}{)}\PY{p}{,}
        \PY{n}{transforms}\PY{o}{.}\PY{n}{RandomResizedCrop}\PY{p}{(}\PY{l+m+mi}{224}\PY{p}{)}\PY{p}{,}
        \PY{n}{transforms}\PY{o}{.}\PY{n}{RandomHorizontalFlip}\PY{p}{(}\PY{p}{)}\PY{p}{,}
        \PY{c+c1}{\PYZsh{}transforms.Resize((224, 224)),\PYZsh{}attention }
        \PY{n}{transforms}\PY{o}{.}\PY{n}{ToTensor}\PY{p}{(}\PY{p}{)}\PY{p}{,}
        \PY{n}{transforms}\PY{o}{.}\PY{n}{Normalize}\PY{p}{(}\PY{p}{[}\PY{l+m+mf}{0.485}\PY{p}{,} \PY{l+m+mf}{0.456}\PY{p}{,} \PY{l+m+mf}{0.406}\PY{p}{]}\PY{p}{,} 
                            \PY{p}{[}\PY{l+m+mf}{0.229}\PY{p}{,} \PY{l+m+mf}{0.224}\PY{p}{,} \PY{l+m+mf}{0.225}\PY{p}{]}\PY{p}{)}
    \PY{p}{]}\PY{p}{)}\PY{p}{,}
    \PY{l+s+s1}{\PYZsq{}}\PY{l+s+s1}{test}\PY{l+s+s1}{\PYZsq{}}\PY{p}{:} \PY{n}{transforms}\PY{o}{.}\PY{n}{Compose}\PY{p}{(}\PY{p}{[}
        \PY{n}{transforms}\PY{o}{.}\PY{n}{ToPILImage}\PY{p}{(}\PY{p}{)}\PY{p}{,}
        \PY{c+c1}{\PYZsh{} transforms.RandomRotation(45),}
        \PY{c+c1}{\PYZsh{} transforms.RandomResizedCrop(224),}
        \PY{c+c1}{\PYZsh{} transforms.RandomHorizontalFlip(),}
        \PY{c+c1}{\PYZsh{}transforms.Resize((224, 224)),\PYZsh{}attention }
        \PY{n}{transforms}\PY{o}{.}\PY{n}{ToTensor}\PY{p}{(}\PY{p}{)}\PY{p}{,}
        \PY{n}{transforms}\PY{o}{.}\PY{n}{Normalize}\PY{p}{(}\PY{p}{[}\PY{l+m+mf}{0.485}\PY{p}{,} \PY{l+m+mf}{0.456}\PY{p}{,} \PY{l+m+mf}{0.406}\PY{p}{]}\PY{p}{,} 
                            \PY{p}{[}\PY{l+m+mf}{0.229}\PY{p}{,} \PY{l+m+mf}{0.224}\PY{p}{,} \PY{l+m+mf}{0.225}\PY{p}{]}\PY{p}{)}
    \PY{p}{]}\PY{p}{)}\PY{p}{,}
\PY{p}{\PYZcb{}}
\end{Verbatim}
\end{tcolorbox}

    \hypertarget{define-train-dataseet-and-dataloader}{%
\subsubsection{Define train dataseet and
dataloader}\label{define-train-dataseet-and-dataloader}}

    \begin{tcolorbox}[breakable, size=fbox, boxrule=1pt, pad at break*=1mm,colback=cellbackground, colframe=cellborder]
\prompt{In}{incolor}{5}{\boxspacing}
\begin{Verbatim}[commandchars=\\\{\}]
\PY{n}{train\PYZus{}dataset} \PY{o}{=} \PY{n}{LazyLoadDataset}\PY{p}{(}\PY{l+s+s2}{\PYZdq{}}\PY{l+s+s2}{./lazydata/}\PY{l+s+s2}{\PYZdq{}}\PY{p}{,} \PY{n}{transform}\PY{o}{=}\PY{n}{data\PYZus{}transforms}\PY{p}{[}\PY{l+s+s1}{\PYZsq{}}\PY{l+s+s1}{train}\PY{l+s+s1}{\PYZsq{}}\PY{p}{]}\PY{p}{)}
\PY{n}{train\PYZus{}loader} \PY{o}{=} \PY{n}{torch}\PY{o}{.}\PY{n}{utils}\PY{o}{.}\PY{n}{data}\PY{o}{.}\PY{n}{DataLoader}\PY{p}{(}\PY{n}{train\PYZus{}dataset}\PY{p}{,} \PY{n}{batch\PYZus{}size}\PY{o}{=}\PY{l+m+mi}{30}\PY{p}{,} \PY{n}{shuffle}\PY{o}{=}\PY{k+kc}{False}\PY{p}{)}
\end{Verbatim}
\end{tcolorbox}

    \hypertarget{training-the-model}{%
\subsection{Training the model}\label{training-the-model}}

    \hypertarget{define-the-training-function}{%
\subsubsection{Define the training
function}\label{define-the-training-function}}

    \begin{tcolorbox}[breakable, size=fbox, boxrule=1pt, pad at break*=1mm,colback=cellbackground, colframe=cellborder]
\prompt{In}{incolor}{6}{\boxspacing}
\begin{Verbatim}[commandchars=\\\{\}]
\PY{k}{def} \PY{n+nf}{train}\PY{p}{(}\PY{n}{epoch}\PY{p}{,} \PY{n}{model}\PY{p}{,} \PY{n}{optimizer}\PY{p}{,} \PY{n}{permute\PYZus{}pixels}\PY{o}{=}\PY{k+kc}{None}\PY{p}{,} \PY{n}{permutation\PYZus{}order}\PY{o}{=}\PY{k+kc}{None}\PY{p}{)}\PY{p}{:}
    \PY{l+s+sd}{\PYZdq{}\PYZdq{}\PYZdq{}}
\PY{l+s+sd}{    Train the model for one epoch}

\PY{l+s+sd}{    Args:}
\PY{l+s+sd}{        epoch (int): current epoch}
\PY{l+s+sd}{        model (nn.Module): model to train}
\PY{l+s+sd}{        optimizer (torch.optim): optimizer to use}
\PY{l+s+sd}{        permute\PYZus{}pixels (function): function to permute the pixels (default: None)}
\PY{l+s+sd}{        permutation\PYZus{}order (1D torch array): order of the permutation (default: None)}
\PY{l+s+sd}{    \PYZdq{}\PYZdq{}\PYZdq{}}
    \PY{n}{model}\PY{o}{.}\PY{n}{train}\PY{p}{(}\PY{p}{)}
    \PY{n}{total\PYZus{}loss} \PY{o}{=} \PY{l+m+mi}{0}
    \PY{k}{for} \PY{n}{batch\PYZus{}idx}\PY{p}{,} \PY{p}{(}\PY{p}{(}\PY{n}{img0}\PY{p}{,} \PY{n}{img1}\PY{p}{,} \PY{n}{img2}\PY{p}{,} \PY{n}{depth}\PY{p}{)}\PY{p}{,} \PY{n}{target}\PY{p}{)} \PY{o+ow}{in} \PY{n+nb}{enumerate}\PY{p}{(}\PY{n}{train\PYZus{}loader}\PY{p}{)}\PY{p}{:}
        \PY{c+c1}{\PYZsh{} send to device}
        \PY{n}{concate} \PY{o}{=} \PY{n}{torch}\PY{o}{.}\PY{n}{cat}\PY{p}{(}\PY{p}{(}\PY{n}{img0}\PY{p}{,} \PY{n}{img1}\PY{p}{,} \PY{n}{img2}\PY{p}{,} \PY{n}{depth}\PY{p}{)}\PY{p}{,} \PY{n}{dim}\PY{o}{=}\PY{l+m+mi}{1}\PY{p}{)}
        \PY{n}{data}\PY{p}{,} \PY{n}{target} \PY{o}{=} \PY{n}{concate}\PY{o}{.}\PY{n}{to}\PY{p}{(}\PY{n}{device}\PY{p}{)}\PY{p}{,} \PY{n}{target}\PY{o}{.}\PY{n}{to}\PY{p}{(}\PY{n}{device}\PY{p}{)}
        
        \PY{c+c1}{\PYZsh{} permute pixels}
        \PY{k}{if} \PY{n}{permute\PYZus{}pixels} \PY{o+ow}{is} \PY{o+ow}{not} \PY{k+kc}{None}\PY{p}{:}
            \PY{n}{data} \PY{o}{=} \PY{n}{permute\PYZus{}pixels}\PY{p}{(}\PY{n}{data}\PY{p}{,} \PY{n}{permutation\PYZus{}order}\PY{p}{)}

        \PY{n}{optimizer}\PY{o}{.}\PY{n}{zero\PYZus{}grad}\PY{p}{(}\PY{p}{)}
        \PY{n}{model} \PY{o}{=} \PY{n}{model}\PY{o}{.}\PY{n}{to}\PY{p}{(}\PY{n}{device}\PY{p}{)}
        \PY{n}{output} \PY{o}{=} \PY{n}{model}\PY{p}{(}\PY{n}{data}\PY{p}{)}
        \PY{n}{lossFn} \PY{o}{=} \PY{n}{nn}\PY{o}{.}\PY{n}{MSELoss}\PY{p}{(}\PY{p}{)}
        \PY{n}{loss} \PY{o}{=} \PY{n}{lossFn}\PY{p}{(}\PY{n}{output}\PY{o}{.}\PY{n}{float}\PY{p}{(}\PY{p}{)}\PY{p}{,} \PY{n}{target}\PY{o}{.}\PY{n}{float}\PY{p}{(}\PY{p}{)}\PY{p}{)}
        \PY{n}{total\PYZus{}loss} \PY{o}{+}\PY{o}{=} \PY{n}{loss}
        \PY{n}{loss}\PY{o}{.}\PY{n}{backward}\PY{p}{(}\PY{p}{)}
        \PY{n}{optimizer}\PY{o}{.}\PY{n}{step}\PY{p}{(}\PY{p}{)}
    
    \PY{c+c1}{\PYZsh{} print average loss}
    \PY{n+nb}{print}\PY{p}{(}\PY{l+s+s1}{\PYZsq{}}\PY{l+s+s1}{Train Epoch: }\PY{l+s+si}{\PYZob{}\PYZcb{}}\PY{l+s+s1}{ }\PY{l+s+se}{\PYZbs{}t}\PY{l+s+s1}{Avg Loss: }\PY{l+s+si}{\PYZob{}:.6f\PYZcb{}}\PY{l+s+s1}{\PYZsq{}}\PY{o}{.}\PY{n}{format}\PY{p}{(}
                \PY{n}{epoch}\PY{p}{,} \PY{n}{total\PYZus{}loss}\PY{o}{/}\PY{n+nb}{len}\PY{p}{(}\PY{n}{train\PYZus{}loader}\PY{o}{.}\PY{n}{dataset}\PY{p}{)}\PY{p}{)}\PY{p}{)}
\end{Verbatim}
\end{tcolorbox}

    \hypertarget{define-the-training-model}{%
\subsubsection{Define the Training
Model}\label{define-the-training-model}}

I use the \texttt{resnet50} model as a base. And modify the \texttt{fc}
and \texttt{conv1} layer to accommendate the training data.

    \begin{tcolorbox}[breakable, size=fbox, boxrule=1pt, pad at break*=1mm,colback=cellbackground, colframe=cellborder]
\prompt{In}{incolor}{7}{\boxspacing}
\begin{Verbatim}[commandchars=\\\{\}]
\PY{c+c1}{\PYZsh{} define the model for training}
\PY{c+c1}{\PYZsh{} use resnet50 as a base}
\PY{n}{model} \PY{o}{=} \PY{n}{resnet50}\PY{p}{(}\PY{n}{weights}\PY{o}{=}\PY{n}{ResNet50\PYZus{}Weights}\PY{o}{.}\PY{n}{DEFAULT}\PY{p}{)}
\PY{n}{model} \PY{o}{=} \PY{n}{model}\PY{o}{.}\PY{n}{to}\PY{p}{(}\PY{n}{device}\PY{p}{)}
\PY{n}{model}\PY{o}{.}\PY{n}{eval}\PY{p}{(}\PY{p}{)}
\PY{n}{model}\PY{o}{.}\PY{n}{float}\PY{p}{(}\PY{p}{)}
\PY{n}{model}\PY{o}{.}\PY{n}{fc} \PY{o}{=} \PY{n}{nn}\PY{o}{.}\PY{n}{Linear}\PY{p}{(}\PY{l+m+mi}{2048}\PY{p}{,} \PY{l+m+mi}{12}\PY{p}{)}
\PY{n}{weight} \PY{o}{=} \PY{n}{model}\PY{o}{.}\PY{n}{conv1}\PY{o}{.}\PY{n}{weight}\PY{o}{.}\PY{n}{clone}\PY{p}{(}\PY{p}{)}
\PY{n}{model}\PY{o}{.}\PY{n}{conv1} \PY{o}{=} \PY{n}{nn}\PY{o}{.}\PY{n}{Conv2d}\PY{p}{(}\PY{l+m+mi}{12}\PY{p}{,} \PY{l+m+mi}{64}\PY{p}{,} \PY{n}{kernel\PYZus{}size}\PY{o}{=}\PY{l+m+mi}{7}\PY{p}{,} \PY{n}{stride}\PY{o}{=}\PY{l+m+mi}{2}\PY{p}{,} \PY{n}{padding}\PY{o}{=}\PY{l+m+mi}{3}\PY{p}{,} \PY{n}{bias}\PY{o}{=}\PY{k+kc}{False}\PY{p}{)}
\PY{k}{with} \PY{n}{torch}\PY{o}{.}\PY{n}{no\PYZus{}grad}\PY{p}{(}\PY{p}{)}\PY{p}{:}
    \PY{n}{model}\PY{o}{.}\PY{n}{conv1}\PY{o}{.}\PY{n}{weight}\PY{p}{[}\PY{p}{:}\PY{p}{,} \PY{p}{:}\PY{l+m+mi}{3}\PY{p}{]} \PY{o}{=} \PY{n}{weight}
    \PY{n}{model}\PY{o}{.}\PY{n}{conv1}\PY{o}{.}\PY{n}{weight}\PY{p}{[}\PY{p}{:}\PY{p}{,} \PY{l+m+mi}{3}\PY{p}{]} \PY{o}{=} \PY{n}{model}\PY{o}{.}\PY{n}{conv1}\PY{o}{.}\PY{n}{weight}\PY{p}{[}\PY{p}{:}\PY{p}{,} \PY{l+m+mi}{0}\PY{p}{]}
\end{Verbatim}
\end{tcolorbox}

    \hypertarget{run-tranning}{%
\subsubsection{Run Tranning}\label{run-tranning}}

I use \texttt{SGD} optimizer and run the model for 40 epoches

    \begin{tcolorbox}[breakable, size=fbox, boxrule=1pt, pad at break*=1mm,colback=cellbackground, colframe=cellborder]
\prompt{In}{incolor}{8}{\boxspacing}
\begin{Verbatim}[commandchars=\\\{\}]
\PY{c+c1}{\PYZsh{} define the optimizer}
\PY{n}{optimizer} \PY{o}{=} \PY{n}{torch}\PY{o}{.}\PY{n}{optim}\PY{o}{.}\PY{n}{SGD}\PY{p}{(}\PY{n}{model}\PY{o}{.}\PY{n}{parameters}\PY{p}{(}\PY{p}{)}\PY{p}{,} \PY{n}{lr}\PY{o}{=}\PY{l+m+mf}{0.01}\PY{p}{,} \PY{n}{momentum}\PY{o}{=}\PY{l+m+mf}{0.9}\PY{p}{)}

\PY{c+c1}{\PYZsh{} train the model for 40 epoches}
\PY{k}{for} \PY{n}{epoch} \PY{o+ow}{in} \PY{n+nb}{range}\PY{p}{(}\PY{l+m+mi}{0}\PY{p}{,} \PY{l+m+mi}{40}\PY{p}{)}\PY{p}{:}
    \PY{n}{train}\PY{p}{(}\PY{n}{epoch}\PY{p}{,} \PY{n}{model}\PY{p}{,} \PY{n}{optimizer}\PY{p}{)}
\end{Verbatim}
\end{tcolorbox}

    \begin{Verbatim}[commandchars=\\\{\}]
Train Epoch: 0  Avg Loss: 29.560263
Train Epoch: 1  Avg Loss: 16.185776
Train Epoch: 2  Avg Loss: 14.481716
Train Epoch: 3  Avg Loss: 13.596376
Train Epoch: 4  Avg Loss: 12.802649
Train Epoch: 5  Avg Loss: 11.612081
Train Epoch: 6  Avg Loss: 10.484567
Train Epoch: 7  Avg Loss: 9.260120
Train Epoch: 8  Avg Loss: 7.155323
Train Epoch: 9  Avg Loss: 4.184078
Train Epoch: 10         Avg Loss: 2.684620
Train Epoch: 11         Avg Loss: 2.191308
Train Epoch: 12         Avg Loss: 1.861231
Train Epoch: 13         Avg Loss: 1.591500
Train Epoch: 14         Avg Loss: 1.384787
Train Epoch: 15         Avg Loss: 1.233111
Train Epoch: 16         Avg Loss: 1.129283
Train Epoch: 17         Avg Loss: 1.041603
Train Epoch: 18         Avg Loss: 0.966110
Train Epoch: 19         Avg Loss: 0.906252
Train Epoch: 20         Avg Loss: 0.852666
Train Epoch: 21         Avg Loss: 0.805656
Train Epoch: 22         Avg Loss: 0.766693
Train Epoch: 23         Avg Loss: 0.731094
Train Epoch: 24         Avg Loss: 0.692568
Train Epoch: 25         Avg Loss: 0.661462
Train Epoch: 26         Avg Loss: 0.629905
Train Epoch: 27         Avg Loss: 0.601338
Train Epoch: 28         Avg Loss: 0.578940
Train Epoch: 29         Avg Loss: 0.556215
Train Epoch: 30         Avg Loss: 0.542907
Train Epoch: 31         Avg Loss: 0.531284
Train Epoch: 32         Avg Loss: 0.511958
Train Epoch: 33         Avg Loss: 0.489195
Train Epoch: 34         Avg Loss: 0.469190
Train Epoch: 35         Avg Loss: 0.448150
Train Epoch: 36         Avg Loss: 0.431364
Train Epoch: 37         Avg Loss: 0.416190
Train Epoch: 38         Avg Loss: 0.400726
Train Epoch: 39         Avg Loss: 0.388667
    \end{Verbatim}

    \hypertarget{export-test-prediction}{%
\subsubsection{Export Test Prediction}\label{export-test-prediction}}

    \begin{tcolorbox}[breakable, size=fbox, boxrule=1pt, pad at break*=1mm,colback=cellbackground, colframe=cellborder]
\prompt{In}{incolor}{9}{\boxspacing}
\begin{Verbatim}[commandchars=\\\{\}]
\PY{k+kn}{import} \PY{n+nn}{pandas} \PY{k}{as} \PY{n+nn}{pd}

\PY{n}{outfile} \PY{o}{=} \PY{l+s+s1}{\PYZsq{}}\PY{l+s+s1}{submission.csv}\PY{l+s+s1}{\PYZsq{}}

\PY{n}{output\PYZus{}file} \PY{o}{=} \PY{n+nb}{open}\PY{p}{(}\PY{n}{outfile}\PY{p}{,} \PY{l+s+s1}{\PYZsq{}}\PY{l+s+s1}{w}\PY{l+s+s1}{\PYZsq{}}\PY{p}{)}

\PY{n}{titles} \PY{o}{=} \PY{p}{[}\PY{l+s+s1}{\PYZsq{}}\PY{l+s+s1}{ID}\PY{l+s+s1}{\PYZsq{}}\PY{p}{,} \PY{l+s+s1}{\PYZsq{}}\PY{l+s+s1}{FINGER\PYZus{}POS\PYZus{}1}\PY{l+s+s1}{\PYZsq{}}\PY{p}{,} \PY{l+s+s1}{\PYZsq{}}\PY{l+s+s1}{FINGER\PYZus{}POS\PYZus{}2}\PY{l+s+s1}{\PYZsq{}}\PY{p}{,} \PY{l+s+s1}{\PYZsq{}}\PY{l+s+s1}{FINGER\PYZus{}POS\PYZus{}3}\PY{l+s+s1}{\PYZsq{}}\PY{p}{,} \PY{l+s+s1}{\PYZsq{}}\PY{l+s+s1}{FINGER\PYZus{}POS\PYZus{}4}\PY{l+s+s1}{\PYZsq{}}\PY{p}{,} \PY{l+s+s1}{\PYZsq{}}\PY{l+s+s1}{FINGER\PYZus{}POS\PYZus{}5}\PY{l+s+s1}{\PYZsq{}}\PY{p}{,} \PY{l+s+s1}{\PYZsq{}}\PY{l+s+s1}{FINGER\PYZus{}POS\PYZus{}6}\PY{l+s+s1}{\PYZsq{}}\PY{p}{,}
         \PY{l+s+s1}{\PYZsq{}}\PY{l+s+s1}{FINGER\PYZus{}POS\PYZus{}7}\PY{l+s+s1}{\PYZsq{}}\PY{p}{,} \PY{l+s+s1}{\PYZsq{}}\PY{l+s+s1}{FINGER\PYZus{}POS\PYZus{}8}\PY{l+s+s1}{\PYZsq{}}\PY{p}{,} \PY{l+s+s1}{\PYZsq{}}\PY{l+s+s1}{FINGER\PYZus{}POS\PYZus{}9}\PY{l+s+s1}{\PYZsq{}}\PY{p}{,} \PY{l+s+s1}{\PYZsq{}}\PY{l+s+s1}{FINGER\PYZus{}POS\PYZus{}10}\PY{l+s+s1}{\PYZsq{}}\PY{p}{,} \PY{l+s+s1}{\PYZsq{}}\PY{l+s+s1}{FINGER\PYZus{}POS\PYZus{}11}\PY{l+s+s1}{\PYZsq{}}\PY{p}{,} \PY{l+s+s1}{\PYZsq{}}\PY{l+s+s1}{FINGER\PYZus{}POS\PYZus{}12}\PY{l+s+s1}{\PYZsq{}}\PY{p}{]}
\PY{n}{preds} \PY{o}{=} \PY{p}{[}\PY{p}{]}

\PY{n}{test\PYZus{}data} \PY{o}{=} \PY{n}{torch}\PY{o}{.}\PY{n}{load}\PY{p}{(}\PY{l+s+s1}{\PYZsq{}}\PY{l+s+s1}{./test/test/testX.pt}\PY{l+s+s1}{\PYZsq{}}\PY{p}{)}
\PY{n}{file\PYZus{}ids} \PY{o}{=} \PY{n}{test\PYZus{}data}\PY{p}{[}\PY{o}{\PYZhy{}}\PY{l+m+mi}{1}\PY{p}{]}
\PY{n}{rgb\PYZus{}data} \PY{o}{=} \PY{n}{test\PYZus{}data}\PY{p}{[}\PY{l+m+mi}{0}\PY{p}{]}
\PY{n}{depth} \PY{o}{=} \PY{n}{test\PYZus{}data}\PY{p}{[}\PY{l+m+mi}{1}\PY{p}{]}
\PY{n}{model}\PY{o}{.}\PY{n}{eval}\PY{p}{(}\PY{p}{)}

\PY{k}{for} \PY{n}{i}\PY{p}{,} \PY{p}{(}\PY{n}{img0}\PY{p}{,} \PY{n}{img1}\PY{p}{,} \PY{n}{img2}\PY{p}{)} \PY{o+ow}{in} \PY{n+nb}{enumerate}\PY{p}{(}\PY{n}{rgb\PYZus{}data}\PY{p}{)}\PY{p}{:}

    \PY{n}{img0} \PY{o}{=} \PY{n}{data\PYZus{}transforms}\PY{p}{[}\PY{l+s+s1}{\PYZsq{}}\PY{l+s+s1}{test}\PY{l+s+s1}{\PYZsq{}}\PY{p}{]}\PY{p}{(}\PY{n}{img0}\PY{p}{)}
    \PY{n}{img1} \PY{o}{=} \PY{n}{data\PYZus{}transforms}\PY{p}{[}\PY{l+s+s1}{\PYZsq{}}\PY{l+s+s1}{test}\PY{l+s+s1}{\PYZsq{}}\PY{p}{]}\PY{p}{(}\PY{n}{img1}\PY{p}{)}
    \PY{n}{img2} \PY{o}{=} \PY{n}{data\PYZus{}transforms}\PY{p}{[}\PY{l+s+s1}{\PYZsq{}}\PY{l+s+s1}{test}\PY{l+s+s1}{\PYZsq{}}\PY{p}{]}\PY{p}{(}\PY{n}{img2}\PY{p}{)}
    
    \PY{n}{data} \PY{o}{=} \PY{n}{torch}\PY{o}{.}\PY{n}{cat}\PY{p}{(}\PY{p}{(}\PY{n}{img0}\PY{p}{,} \PY{n}{img1}\PY{p}{,} \PY{n}{img2}\PY{p}{,} \PY{n}{depth}\PY{p}{[}\PY{n}{i}\PY{p}{]}\PY{p}{)}\PY{p}{,} \PY{n}{dim}\PY{o}{=}\PY{l+m+mi}{0}\PY{p}{)}
    \PY{n}{data} \PY{o}{=} \PY{n}{torch}\PY{o}{.}\PY{n}{unsqueeze}\PY{p}{(}\PY{n}{data}\PY{p}{,} \PY{l+m+mi}{0}\PY{p}{)}
    \PY{n}{output} \PY{o}{=} \PY{n}{model}\PY{p}{(}\PY{n}{data}\PY{o}{.}\PY{n}{to}\PY{p}{(}\PY{n}{device}\PY{p}{)}\PY{p}{)} \PY{o}{/} \PY{l+m+mi}{1000}
    \PY{n}{preds}\PY{o}{.}\PY{n}{append}\PY{p}{(}\PY{n}{output}\PY{p}{[}\PY{l+m+mi}{0}\PY{p}{]}\PY{o}{.}\PY{n}{cpu}\PY{p}{(}\PY{p}{)}\PY{o}{.}\PY{n}{detach}\PY{p}{(}\PY{p}{)}\PY{o}{.}\PY{n}{numpy}\PY{p}{(}\PY{p}{)}\PY{p}{)}

\PY{n}{df} \PY{o}{=} \PY{n}{pd}\PY{o}{.}\PY{n}{concat}\PY{p}{(}\PY{p}{[}\PY{n}{pd}\PY{o}{.}\PY{n}{DataFrame}\PY{p}{(}\PY{n}{file\PYZus{}ids}\PY{p}{)}\PY{p}{,} \PY{n}{pd}\PY{o}{.}\PY{n}{DataFrame}\PY{o}{.}\PY{n}{from\PYZus{}records}\PY{p}{(}\PY{n}{preds}\PY{p}{)}\PY{p}{]}\PY{p}{,} \PY{n}{axis} \PY{o}{=} \PY{l+m+mi}{1}\PY{p}{,} \PY{n}{names} \PY{o}{=} \PY{n}{titles}\PY{p}{)}
\PY{n}{df}\PY{o}{.}\PY{n}{columns} \PY{o}{=} \PY{n}{titles}
\PY{n}{df}\PY{o}{.}\PY{n}{to\PYZus{}csv}\PY{p}{(}\PY{n}{outfile}\PY{p}{,} \PY{n}{index} \PY{o}{=} \PY{k+kc}{False}\PY{p}{)}
\PY{n+nb}{print}\PY{p}{(}\PY{l+s+s2}{\PYZdq{}}\PY{l+s+s2}{Written to csv file }\PY{l+s+si}{\PYZob{}\PYZcb{}}\PY{l+s+s2}{\PYZdq{}}\PY{o}{.}\PY{n}{format}\PY{p}{(}\PY{n}{outfile}\PY{p}{)}\PY{p}{)}
\end{Verbatim}
\end{tcolorbox}

    \begin{Verbatim}[commandchars=\\\{\}]
Written to csv file submission.csv
    \end{Verbatim}

    \hypertarget{experiment-result}{%
\section{Experiment Result}\label{experiment-result}}

    The highest public score of the prediction on Kaggle is 0.00372.\\
However, every time the model run, the score varies from 0.00372 to
0.00628.

    \hypertarget{discussion}{%
\section{Discussion}\label{discussion}}

    \hypertarget{loading}{%
\subsection{Loading}\label{loading}}

    \begin{itemize}
\tightlist
\item
  Lazy loading can save a lot of time
\item
  It is also memory friendly
\item
  Large batch size can result in memory shortage so decreases batch size
  to save GPU
\end{itemize}

    \hypertarget{normalization}{%
\subsection{Normalization}\label{normalization}}

    \begin{itemize}
\tightlist
\item
  Normalization on RGB images can improve the performance of the model
  tremendously
\item
  Possibly because some of the data are very large compare to the other
  ones
\item
  It is very useful to multiple the Y by 1000 when loading the data as
  the Y is smaller than X
\item
  Transform depth image doesn't improve the performance much
\end{itemize}

    \hypertarget{model}{%
\subsection{Model}\label{model}}

    \begin{itemize}
\tightlist
\item
  Using pre-trained model(like resnet50 in this case) can provide a good
  start
\item
  Need to modify the fc layer of the model to accomadate the data
\item
  SGD optimizer in my case perform better than Adam optimizer
\item
  Running multiple epoches can improve the model tremendously
\end{itemize}

    \hypertarget{future-work}{%
\section{Future Work}\label{future-work}}

    \begin{itemize}
\tightlist
\item
  Trying different normalization methods to see if we can achieve a
  better result
\item
  Trying more optimizers or modifying optimizers' parameters like
  learning rate
\item
  Using other pre-trained or self-defined models
\item
  Running the model for more epoches
\end{itemize}

    


    % Add a bibliography block to the postdoc
    
    
    
\end{document}
